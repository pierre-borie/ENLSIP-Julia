

%   Generate the environment for the abstract:
\newcommand\summaryname{Abstract}
\newenvironment{Abstract}%
    {\small\begin{center}%
    \bfseries{\summaryname} \end{center}}

%   Generate the environment for the Résumé:
\newcommand\nomresume{Résumé}
\newenvironment{Resume}%
    {\small\begin{center}%
    \bfseries{\nomresume} \end{center}}

% Insert Résumé
\begin{Resume}
    \begin{changemargin}{1cm}{1cm}
        Afin de prévoir efficacement la demande en électricité à court terme, Hydro-Québec, principal producteur d’électricité du Québec, 
        calibre ses modèles de prévision grâce à l'utilisation d'ENLSIP, algorithme d'optimisation écrit en Fortran77.  
        Néanmoins, sa maintenance est désormais complexe à assurer. Afin d'améliorer sa fiabilité, 
        nous en proposons une nouvelle implémentation réalisée dans le langage Julia, langage informatique moderne conçu pour le calcul scientifique haute 
        performance et les sciences des données. 
        Cette dernière s'accompagne d'une modélisation de la méthode d'optimisation utilisée ainsi que de tests effectués sur des problèmes de moindres carrés 
        pour en évaluer les performances. Sont ensuite explorées certaines pistes d'amélioration. 
     
    \end{changemargin}
\end{Resume}

\vspace{1cm}
        
\textbf{Mots-clés:} Optimisation non linéaire sous contraintes, moindres carrés, calibration de modèle, demande en énergie, langage de programmation Julia.

\vspace{2cm}
% Insert bastract
\begin{Abstract}
\begin{changemargin}{1cm}{1cm}
    In order to predict efficiently short term demand, the main electricity producer in Quebec, Hydro-Quebec, calibrates its own previsions models by using ENLSIP,
    an algorithm written in Fortran77 harder to maintain correctly nowadays. This report exposes a new implementation made in the Julia programming language. 
    A modelisation of the optimisation method used is detailled. The performances are also evaluated by running the algorithm on least squares test problems. Finally,
    some ideas of improvement are discussed.
\end{changemargin}
\end{Abstract}

\vspace{1cm}

\textbf{Keywords: } Constrained nonlinear programming, least squares, model calibrating, energy demand, Julia programming language.
