\chapter{Travail de recherche et implémentation}\label{Travail}
\markright{Travail de recherche et implémentation}


Ce chapitre présente d'abord le fruit de l'étude des travaux de~\citet{lindwedin88}; sont ensuite traités les différents 
aspects de l'implémentation de cet algorithme en Julia résultant du travail de recherche réalisé.

\section{Description de la méthode d'ENLSIP}\label{description_methode}

L'analyse approfondie des articles de~\citet{lindwedin88}~\cite{lindstromwedin1984} sur la méthode d'ENLSIP, ainsi que du code source de l'algorithme écrit en Fortran77 
par ces derniers, m'a permis de comprendre le fonctionnement de cette méthode d'optimisation. Cette partie, réalisée sur recommandation de mon maître de stage, 
en relate les principaux aspects théoriques. 
On se place dans le cadre où le problème d'optimisation à résoudre est celui présenté en~\eqref{pb_general}.

\subsection{Principe général}

Tout d'abord, il s'agit d'une méthode itérative avec longueur de pas. On rappelle que cela consiste, en partant d'un point de départ $x_0$, à construire une suite
d'itérés $(x_k)_{K\in \mathbb{N}}$ convergeant vers la solution du problème.

\section{Implémentation en Julia}

Dès lors que je suis parvenu à modéliser et comprendre les aspects théoriques et pratiques de l'algorithme ENLSIP en Fortran77, j'ai pu réalisé son implémentation en Julia.