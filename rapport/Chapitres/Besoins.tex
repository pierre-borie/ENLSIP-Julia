\chapter{Besoins et organisation du projet}\label{Besoins}
\markright{Besoins et organisation du projet}

\section{Besoins}

Avant d'expliciter le travail effectué lors de ce projet, il convient d'aborder ce qui est attendu de la part de l'UdeM 
et d'\HQ.

\subsection*{Livrables et travail à effectuer}\label{livrables}

Dans un premier temps, certains livrables ont été définis avec mes collaborateurs, dont mon maître de stage. 
Les principales attentes concernent l'implémentation de l'algorithme ENLSIP en Julia.
Le code source de ce dernier doit en effet être fourni à travers différents fichiers Julia (format .jl) nécessaires
pour l'exécution de l'algorithme. Ce travail ayant vocation à être disponible en open source sous la forme d'une librairie Julia, une documentation en anglais
indiquant comment utiliser le solveur et consultable via un fichier HTML est également attendue. 

Afin de s'assurer de l'efficacité de l'implémentation réalisée, différents tests sur des problèmes de moindres carrés seront réalisés tout au long de son développement.
Ces derniers sont effectués et consultables par des fichiers notebooks Jupyter (format .ipynb), type de fichier compatible avec 
l'exécution du code Julia. La modélisation des problèmes expérimentés ainsi que les résultats attendus doivent figurer sur ces fichiers de tests. 

Les problèmes testés seront de deux catégories:
\begin{itemize}
    \item problèmes de programmation non linéaire documentés s'inscrivant dans un cadre purement mathématique;
    \item problèmes formés à partir des modèles et données réelles utilisés par \HQ\ en contexte opérationnel.
\end{itemize}

Le premier type de problèmes a pour vocation de vérifier le bon déroulement de l'algorithme et sa concordance avec la version
codée en Fortran77.

Le dessein derrière le second type de problèmes est de tester l'algorithme dans des cas d'utilisation
correspondant au contexte industriel auquel l'unité de prévision de la demande d'\HQ\ fait face quotidiennement. Ces problèmes-ci seront d'avantage mis à contribution
lorsque l'implémentation d'ENLSIP en Julia sera terminée, afin d'amorcer le remplacement de l'algorithme en Fortran77 par la nouvelle version en Julia. 
Les différents résultats obtenus à travers des tests effectués sont décrits dans la section~\ref{implementation:resultats}.

Le dernier livrable attendu, requis par Fabian Bastin mon maître de stage, concerne plus particulièrement la méthode d'optimisation 
employée dans ENLSIP. En effet, la description mathématique de celle-ci n'est pas intégralement documentée 
par ses auteurs~\citet{lindwedin88}. Or, en vue de moderniser la méthode implémentée dans cet algorithme, il est important
d'en comprendre tous les aspects théoriques afin de mieux cibler les points qui peuvent être améliorés. Une documentation avec la formulation mathématique des 
différents éléments de la méthode d'optimisation d'ENLSIP est donc attendue. Le fruit de ce travail est présentée au chapitre~\ref{Implementation}.

\subsection*{Résultats et performances attendus}\label{besoins:resultats}

Ensuite, la finalité de ce projet étant de produire un algorithme devant être utilisé en contexte industriel, des résultats et performances sont attendus de la part
d'\HQ. 

L'objectif premier derrière le passage à Julia est, pour \HQ, de fiabiliser plusieurs de ses outils de prévision de la demande en électricité. Cela passe par la retranscription
 exacte de la méthode usitée dans ENLSIP codée originellement Fortran77 tout en tirant parti des avantages de programmation conférés par le nouveau langage Julia, plus moderne et moins complexe à maintenir 
 que le Fortran77. L'amélioration de l'algorithme et de la méthode utilisée à proprement parler ne vient que dans un second temps. Il est donc primordial de d'abord 
 produire un outil renvoyant des résultats identiques, à la précision machine près, à ceux obtenus avec la version Fortran77. 

Enfin, les résultats en termes de temps de calcul entre les deux versions doivent être similaires ou réduits afin que la transition au Julia ne se fasse pas au prix d'une 
 dégradation des performances. Puisqu'il s'agit d'un optimiseur en appui à des outils opérationnels de prévision de la demande court terme, les temps d'exécution
 sont soumis à des contraintes de temps de calcul, en particulier pour les cas rencontrés en contexte opérationnel. Celles-ci sont de cet ordre:

 \begin{itemize}
     \item de $10$ à $30$ secondes sur des problèmes de petite taille (i.e. jusqu'à cinq paramètres);
     \item de $10$ à $15$ minutes pour des problèmes de grande taille, soit issus du contexte opérationnel (i.e. de l'ordre de plusieurs centaines de paramètres).
 \end{itemize}

 \section{M\'ethode de travail}

Ce projet ayant été réalisé en télé-travail permanent depuis mon domicile à Paris, l'éloignement géographique avec mes collaborateurs, 
tous résidents au Québec, a donc d\^u \^etre pris en compte dans son organisation. \'Etant donné que j'étais seul à travailler sur la retranscription 
en Julia de l'algorithme en Fortran77, cela n'a pas été un réel obstacle pour avancer dans mes recherches. 

\subsection*{Développement itératif}

Vu la complexité la complexité que présente l'algorithme ENLSIP, il a fallu instaurer une stratégie de développement avec mon maître de stage.

L'idée a été d'ajouter chacune des fonctionnalités de la méthode, décrites au chapitre~\ref{Implementation}, de façon itérative, ceci afin de complexifier
au fur et à mesure l'implémentation en Julia. Des entrevues avec mon maître de stage se sont tenues de façon hebdomadaire. Ces dernières étaient dédiées d'abord 
à la présentation de mon travail de la semaine passée, puis aux points sur lesquels travailler pour la semaine à venir. Cela a permis de s'adapter aux difficultés
rencontrées au cours du développement de l'algorithme. Afin de garder une traçabilité de toutes les versions intermédiaires de mon travail, un répertoire github
a également été mis en place.

\`A chaque nouvel ajout d'une fonctionnalité, mon implémentation a été testée sur les problèmes de tests exposés à la section~\ref{implementation:resultats}. 
Cela a régulièrement permis de repérer quels éléments devaient être améliorés en priorité pour la prochaine version.

\subsection*{Collaboration avec Hydro-Québec}

Ensuite, comme mon stage s'inscrit dans un projet plus global de modernisation des outils de prévision de la demande d'\HQ, certains membres de l'unité Prévisions de la demande
ont également été des parties prenantes de mon PFE. Ces dernières m'ont aidé à accéder à des environnements de travail me permettant de tester mon implémentation 
d'ENLSIP sur des machines virtuelles et infrastructures proches de celles d'\HQ\ en contexte opérationnel. Une présentation de mon travail a eu lieu de fa\c con régulière 
avec les membres d'\HQ impliqués dans le projet également sur une base hebdomadaire.

Cette collaboration avait également pour objectif de mettre en place différents tests de mon algorithme sur des jeux de données réelles et avec des modèles de 
prévision utilisés par \HQ. Néanmoins, cela s'est jusqu'à présent avéré difficile à mettre en place. En effet, les modèles étant relativement complexes et 
codés en Fortran, une passerelle entre les langages Fortran et Julia a dû être mise en place afin de pouvoir appeler ces fonctions en Fortran depuis un 
script en Julia. Ce travail est néanmoins fortement avancé et sera mis à contribution lors de travaux futurs sur le projet. 